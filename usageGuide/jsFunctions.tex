\documentclass[../main.tex]{subfiles}
\begin{document}

\section{Writing Your Own JavaScript Functions}\label{sec:jsFunctions}

One of the main advantages of this framework is that you are able to write a function once 
and then deploy it to any supported platform without additional modification afterwards.
To this end, Node.js and its module semantics are used. A typical function can look like this:

\begin{minted}[frame=single]{js}
  const lib = require('@faastermetrics/lib')
  module.exports = lib.serverless.rpcHandler(async (event, ctx) => {
    return {result: 'Hello World'}
  })
\end{minted}

As can be seen, our FaaSterMetrics library is used universally across the project. We also wrote an asynchronous JavaScript function that returns a JavaScript object with a ‘Hello World’ string. This function is passed to the so-called ‘rpcHandler’ which then gets exported. 

\subsection{Function Naming Convention}

Index js
Of course requiring other modules/ files possible

\subsection{Function Wrapper}
koa.js (footnote)

`rpcHandler'
Main function handler, used nearly everywhere, true to FaaS paradigm
What are event, ctx

Router
Sometimes routes needed (e.g.\@ in our Webshop frontend)
(Ask Emily for route fix explanation)

\subsection{Database Access}
Redis DB for experiment
Hosted on AWS EC2

\subsection{Calling Other FaaS Functions}
Assumption: unique name


\subsection{Custom Log Output}

\subsection{Endpoint /call and JSON object passing}


\end{document}

