\documentclass[../main.tex]{subfiles}
\begin{document}

\section{Implementing Your Own JavaScript Functions}\label{sec:jsFunctions}

One of the main advantages of this framework is the ability to write a function once 
and then deploy it to any supported platform without modification afterwards.
To this end, Node.js and its module semantics are used. A typical function may look like this:

\begin{figure}[H]
\begin{tcolorbox}
\begin{minted}{js}
  const lib = require('@faastermetrics/lib')
  module.exports = lib.serverless.rpcHandler(async (event, ctx) => {
    return {result: 'Hello World'}
  })
\end{minted}
\end{tcolorbox}
\caption{A ``Hello World'' function in the FaaSterMetrics framework}%
\label{fig:fmHelloWorld}
\end{figure}

As can be seen, our FaaSterMetrics library is used universally across the project. 
We~also wrote an asynchronous JavaScript function that returns a JavaScript object with a \mintinline{js}{'Hello World'} string. 
It takes two arguments and is then passed to a so-called `rpcHandler' which gets exported (\Cref{ssub:rpcHandler}). 

\newpage{}%TODO review
\subsection{Function Naming Convention}

Each FaaS function of an experiment is found in a file named \texttt{index.js} which has its own directory within the experiment's functions folder.
Thus, the path to the function code should end as follows (see also -TODO:\@ redacted-):%
\begin{tcolorbox}
  \texttt{[\ldots]/experiments/<experiment\_name>/functions/<function\_name>/index.js}
\end{tcolorbox}

Of course, it is possible to create additional files or directories within this structure in order to use them in your JavaScript code.
This can e.g.\@ be useful to structure code, but also if some functions or data sets shall be used across different FaaS functions.
In such cases, we suggest to utilize Node.js module semantics as shown in \Cref{fig:dividingDataAndFunctionCodeNodeJS}.

\begin{figure}
\begin{tcolorbox}[left=3mm, right=3mm, top=3mm]
  \begin{tcolorbox}[titleDetachedStyle, title=\texttt{{[\ldots]/exampleFunction/data.js}}]
  \begin{minted}{js}
  const prices = {
    umbrella: 20,
    coat: 60,
    flowerPot: 5
  }

  module.exports = { prices }
  \end{minted}
  \end{tcolorbox}

  \begin{tcolorbox}[titleDetachedStyle, title=\texttt{{[\ldots]/exampleFunction/index.js}}]
  \begin{minted}{js}
  const { prices } = require('./data')
  const lib = require('@faastermetrics/lib')

  module.exports = lib.serverless.rpcHandler(async _ => {
    const totalCost = Object.values(prices).reduce((a,b) => a+b)
    return { totalCost }
  })
  \end{minted}
  \end{tcolorbox}

\end{tcolorbox}
\caption{Using Node.js module semantics to split some data and a function using it into separate files.}%
\label{fig:dividingDataAndFunctionCodeNodeJS}
\end{figure}

\subsection{Function Wrapper}\label{sub:jsFunctionWrapper}

In order to use the capabilities of our framework, user functions always have to be exported and wrapped.
This is already depicted in the examples above where the user function is exported within our library's `rpcHandler'.
It allows us to integrate them into our unified deployment process and to add our own control data (e.g.\@ for later analysis).
Internally, \texttt{koajs}\footnote{\url{https://koajs.com}} is used as middleware. See -TODO:\@ redacted- for details.

\subsubsection{`rpcHandler'}%
\label{ssub:rpcHandler}
This is the main (and usually suggested) function wrapper of the FaaSterMetrics library (`lib'), 
accessible via \mintinline{js}{lib.serverless.rpcHandler} (see \Cref{fig:fmHelloWorld}).
It is true to the standard FaaS paradigm, as it allows for simple functions 
that do one single calculation and are called solely via an HTTP(S)-request.

It is also easy to use, since it just needs to be given the asynchronous user function as parameter. 
That user function, in turn, may use two arguments: \mintinline{js}{event} and \mintinline{js}{ctx}.
\mintinline{js}{event} contains the payload object (which may be empty, of course), whereas \mintinline{js}{ctx} makes certain 
library operations accessible to the user (see \Cref{sub:functionDBAccess,sub:functionCallingOther,sub:functionCustomLogs}).
The user function is expected to return a JavaScript object

The default exposed route in order to call functions wrapped by rpcHandlers is 
\texttt{<platform\_endpoint>/<function\_name>/call}.
See also \Cref{fig:rpcHandlerFullFunctionalityExample} for a complete rpcHandler example.

\subsubsection{`router'}%
\label{ssub:functionRouter}

Sometimes it may become necessary to get more control over internals than the rpcHandler allows.
For instance, if you are developing some website frontend, you probably want to be able to configure multiple HTTP routes per function by yourself
instead of relying on a single \texttt{/call} endpoint for each function.
Also you may sometimes be interested in writing functions whose input or output is not just a JavaScript object (or JSON, respectively).

In such cases you can use the FaaSterMetrics library's router wrapper, 
accessible via \mintinline{js}{lib.serverless.router}.
While this does allow you to use the full functionality of a \texttt{koa router}\protect\footnotemark,
it also adds an often unneeded layer of complexity and is not meant to be used universally within an application.
\footnotetext{See \url{https://github.com/ZijianHe/koa-router} in specific as well as
\url{https://koajs.com} in general for information on koa routers.}

As the name implies, this wrapper needs to be given an asynchronous \texttt{koa router}
instead of a simple function as main argument.
Thus, within your routed user functions is also a \texttt{koa} context (usually named \mintinline{js}{ctx})
accessible. 
In particular, a function's request object can be accessed via \mintinline{js}{ctx.request.body}
and its response object can be written into \mintinline{js}{ctx.body}.

See also \Cref{fig:routerFullFunctionalityExample} for a complete rpcHandler example.

\subsection{Calling Other FaaS Functions}\label{sub:functionCallingOther}

In order to build applications out of FaaS functions and later analyse their performance, it is crucial for one function to be able to call another one.
While doing this our framework needs to generate logs containing all kinds of relevant information about those functions and their calling process. 

Within the wrapped user function, this can be achieved by using a member function of the passed \mintinline{js}{ctx} object.

When an rpcHandler is used as function wrapper, other functions are simply called via
\mintinline{js}{ctx.call('functionName', functionPayload)},
where the used arguments represent name and event object (i.e.\@ payload) of the called function.
Within a router, use \mintinline{js}{ctx.lib.call('functionName', functionPayload)} instead.
Also, since functions are typically asynchronous, it is often necessary to \mintinline{js}{await} those calls.

Complete examples of user functions which call other functions are shown in 
\Cref{fig:rpcHandlerFullFunctionalityExample,fig:routerFullFunctionalityExample}.

\subsection{Database Access}\label{sub:functionDBAccess}

If you have configured a redis database for an experiment (see ---TODO:\@ redacted--- for how to do this),
this database becomes available to a function by passing an option object to the respective function wrapper.
Assuming that database is called \mintinline{json}{"redis"} within the \texttt{experiment.json} config file,
this means a JavaScript object of the form \mintinline{js}{{db: 'redis'}} has to be given to the used function wrapper
as first argument --- additionally to the function or koa router to be wrapped, which then embodies the second argument.
The database can then be accessed via the \mintinline{js}{ctx} argument within that function.

Since redis is used, the database has the form of a string-based key-value store. 
Thus, keys should always be simple strings. 
Values can be strings, arrays or JavaScript objects. 

Writing or updating a key-value pair from a function wrapped by an rpcHandler is done via \mintinline{js}{ctx.db.set(key, value)}
and retrieving a value is done via \mintinline{js}{ctx.db.get(key)}.
Whenever a router is used as function wrapper, these two instructions should be exchanged for
\mintinline{js}{ctx.lib.db.set(key, value)} and \mintinline{js}{ctx.lib.db.get(key)}, respectively.
Also,~since user functions are usually asynchronous, it is mostly necessary to \mintinline{js}{await} their results.

Complete examples of user functions that also use a database can be seen in 
\Cref{fig:rpcHandlerFullFunctionalityExample,fig:routerFullFunctionalityExample}.

\subsection{Custom Log Output}\label{sub:functionCustomLogs}

While the FaaSterMetrics library is usually sufficient for all logging purposes (see TODO:\@ ---redacted---),
you may sometimes feel the need to introduce your own log statements and still get the usual timestamps, etc.\@
for future analysis.
In these situations, you can just use the \mintinline{js}{ctx.log('string_to_be_logged')} statement,
where \mintinline{js}{ctx} is again the respective argument of your function.
Within a function wrapped by a router, \mintinline{js}{ctx.lib.log('string_to_be_logged')} has to be used instead.

\subsection{Other Helper Functions}\label{sub:helperFunctions}

If you ever want to write a function that changes its behaviour depending on where it is deployed,
we have introduced simple helper functions that return truthy if and only if 
the function is currently deployed on the respective provider:
\begin{itemize}
  \item \mintinline{js}{lib.helper.isLambda()}
  \item \mintinline{js}{lib.helper.isGoogle()}
  \item \mintinline{js}{lib.helper.isAzure()}
  \item \mintinline{js}{lib.helper.isTinyfaas()}
  \item \mintinline{js}{lib.helper.isOpenfaas()}
  \item \mintinline{js}{lib.helper.isOpenwhisk()}
\end{itemize}
However their usage is strongly discouraged, 
since it almost always contradicts the project's design principles to write platform-specific code.

If you ever need to use some random ID in the form of a string,
you can create it with \mintinline{js}{lib.helper.generateRandomID()} (which uses \mintinline{js}{Math.random()} internally).

\begin{figure}
\begin{tcolorbox}
\begin{minted}{js}
  const lib = require('@faastermetrics/lib')
  module.exports = lib.serverless.rpcHandler(
    { db: 'redis' }, async (event, ctx) => {
      const { a, b } = event
  
      // Calling other functions to do calculations
      const square = await ctx.call('multiply', { a: a, b: a })
      const result = await ctx.call('add', { a: square, b: b })
  
      // Convert result to string and store within database
      const str = `a squared plus b is ${result}.`
      await ctx.db.set('ID', str)

      // Read out of database and create a custom log out of it
      await ctx.log(await ctx.db.get('ID'))

      return { result: str }
    }
  )
\end{minted}
\end{tcolorbox}
\caption{%
  A simple function showcasing all library capabilities within an rpcHandler.
  We expect that there are two other deployed functions `add' and `multiply',
  which return the sum and product of two numbers \mintinline{js}{a} and \mintinline{js}{b}, respectively.
  After deployment, the function itself can trivially be called by executing the command in \Cref{fig:curlExampleCall}.
}%
\label{fig:rpcHandlerFullFunctionalityExample}
\end{figure}

\begin{figure}
\begin{tcolorbox}
\begin{minted}{js}
  const lib = require('@faastermetrics/lib')
  module.exports = lib.serverless.router(
    { db: 'redis' }, async router => {
      router.post('/call', async (ctx, next) => {
        const { a, b } = ctx.request.body
    
        // Calling other functions to do calculations
        const square = await ctx.lib.call('multiply', { a: a, b: a })
        const result = await ctx.lib.call('add', { a: square, b: b })
   
        // Convert result to string and store within database
        const str = `a squared plus b is ${result}.`
        await ctx.lib.db.set('ID', str)
  
        // Read out of database and create a custom log out of it
        await ctx.lib.log(await ctx.db.get('ID'))
  
        ctx.type = 'application/json'
        ctx.body = { result: str }
      }
    }
  )
\end{minted}
\end{tcolorbox}
\caption{%
  The same functionality as in \Cref{fig:rpcHandlerFullFunctionalityExample}, 
  but using the router function wrapper instead of the rpcHandler.
  It can also be called by executing the command in \Cref{fig:curlExampleCall} 
  after it has been deployed.
}%
\label{fig:routerFullFunctionalityExample}
\end{figure}

\begin{figure}
  \begin{tcolorbox}[left=-2.0mm, top=3mm, bottom=3mm]
  \hfuzz=5pt
  \begin{verbatim}
  curl --header "Content-Type: application/json" --data '{"a":10,"b":5}' 
  <platform_endpoint>/<function_name>/call\end{verbatim}
  \end{tcolorbox}
  \caption{Example \texttt{curl} shell command for calling the functions of 
  \Cref{fig:rpcHandlerFullFunctionalityExample,fig:routerFullFunctionalityExample}}
  Of course, \texttt{<platform\_endpoint>/<function\_name>} has to be replaced according to 
  where the function is deployed and how it is named.\label{fig:curlExampleCall}
\end{figure}


\end{document}

