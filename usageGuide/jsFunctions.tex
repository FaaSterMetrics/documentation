\documentclass[../main.tex]{subfiles}
\begin{document}

\section{Writing Your Own JavaScript Functions}\label{sec:jsFunctions}

One of the main advantages of this framework is the ability to write a function once 
and then deploy it to any supported platform without modification afterwards.
To this end, Node.js and its module semantics are used. A typical function may look like this:

\begin{figure}[H]
\begin{tcolorbox}
\begin{minted}{js}
  const lib = require('@faastermetrics/lib')
  module.exports = lib.serverless.rpcHandler(async (event, ctx) => {
    return {result: 'Hello World'}
  })
\end{minted}
\end{tcolorbox}
\caption{A ``Hello World'' function in the FaaSterMetrics framework}%
\label{fig:fmHelloWorld}
\end{figure}

As can be seen, our FaaSterMetrics library is used universally across the project. 
We~also wrote an asynchronous JavaScript function that returns a JavaScript object with a \mintinline{js}{'Hello World'} string. 
It takes two arguments and is then passed to a so-called `rpcHandler' which gets exported (\Cref{ssub:rpcHandler}). 

\newpage{}%TODO review
\subsection{Function Naming Convention}

Each FaaS function of an experiment is found in a file named index.js which has its own directory within the experiment's functions folder.
Thus, the path to the function code should end as follows (see also -TODO:\@ redacted-):%
\begin{tcolorbox}
  \texttt{[\ldots]/experiments/<experiment\_name>/functions/<function\_name>/index.js}
\end{tcolorbox}

Of course, it is possible to create additional files or directories within this structure in order to use them in your JavaScript code.
This can e.g.\@ be useful to structure code, but also if some functions or data sets shall be used across different FaaS functions.
In such cases, we suggest to utilize Node.js module semantics as shown in \Cref{fig:dividingDataAndFunctionCodeNodeJS}.

\begin{figure}
\begin{tcolorbox}[left=3mm, right=3mm, top=3mm]
  \begin{tcolorbox}[enhanced, title={\texttt{[\ldots]/exampleFunction/data.js}}, attach boxed title to bottom left, boxed title style={sharp corners}]
  \begin{minted}{js}
  const prices = {
    umbrella: 20,
    coat: 60,
    flowerPot: 5
  }

  module.exports = { prices }
  \end{minted}
  \end{tcolorbox}

  \begin{tcolorbox}[enhanced, title=\texttt{{[\ldots]/exampleFunction/index.js}}, attach boxed title to bottom left, boxed title style={sharp corners}]
  \begin{minted}{js}
  const { prices } = require('./data')
  const lib = require('@faastermetrics/lib')

  module.exports = lib.serverless.rpcHandler(async _ => {
    const totalCost = Object.values(prices).reduce((a,b) => a+b)
    return { totalCost }
  })
  \end{minted}
  \end{tcolorbox}

\end{tcolorbox}
\caption{Using Node.js module semantics to split some data and a function using it into separate files.}%
\label{fig:dividingDataAndFunctionCodeNodeJS}
\end{figure}

\subsection{Function Wrapper}\label{sub:jsFunctionWrapper}

In order to use the capabilities of our framework, user functions always have to be exported and wrapped.
This is already depicted in the examples above where the user function is exported within our library's `rpcHandler'.
It allows us to integrate them into our unified deployment process and to add our own control data (e.g.\@ for later analysis).
Internally, \texttt{koajs}\footnote{\url{https://koajs.com}} is used as middleware. See -TODO:\@ redacted- for details.

\subsubsection{`rpcHandler'}%
\label{ssub:rpcHandler}
This is the main (and usually suggested) function wrapper of the FaaSterMetrics library (`lib'), 
accessible via \mintinline{js}{lib.serverless.rpcHandler} (see \Cref{fig:fmHelloWorld}).
It is true to the standard FaaS paradigm, as it allows for simple functions 
that do one single calculation and are called solely via an HTTP(S)-request.

It is also easy to use, since it just needs to be given the asynchronous user function as parameter. 
That user function, in turn, may use two arguments: \mintinline{js}{event} and \mintinline{js}{ctx}.
\mintinline{js}{event} contains the payload object (which may be empty, of course), whereas \mintinline{js}{ctx} makes certain 
library operations accessible to the user (see \Cref{sub:functionDBAccess,sub:functionCallingOther,sub:functionCustomLogs}).
The user function is expected to return a JavaScript object

The default exposed route in order to call functions wrapped by rpcHandlers is 
\texttt{<platform\_endpoint>/<function\_name>/call}.
See also TODO:\@ FIGURE for complete rpcHandler example.

\subsubsection{`router'}%
\label{ssub:functionRouter}

Sometimes it may become necessary to get more control over internals than the rpcHandler allows.
For instance, if you are developing some website frontend, you probably want to be able to configure multiple routes per function by yourself
instead of relying on a single \texttt{/call} endpoint for each function.
In such cases you can use the FaaSterMetrics library's router wrapper, 
accessible via \mintinline{js}{lib.serverless.router}.
While this does allow you to use the full functionality of a \texttt{koa router}\protect\footnotemark,
it also adds an often unneeded layer of complexity.
\footnotetext{See \url{https://github.com/ZijianHe/koa-router} in specific as well as
\url{https://koajs.com} in general for information on koa routers.}

TODO

\subsection{Database Access}\label{sub:functionDBAccess}
Redis DB for experiment
Hosted on AWS EC2
Supported types

\subsection{Calling Other FaaS Functions}\label{sub:functionCallingOther}
Assumption: unique name


\subsection{Custom Log Output}\label{sub:functionCustomLogs}
timestamps etc obviously done by lib

\subsection{Other Helper Functions}\label{sub:helperFunctions}

\begin{figure}
\begin{tcolorbox}
\begin{minted}{js}
  const lib = require('@faastermetrics/lib')
  module.exports = lib.serverless.rpcHandler(
    { db: 'redis' }, async (event, ctx) => {
      const { a, b } = event
  
      // Calling other functions to do calculations
      const square = await ctx.call('multiply', { a: a, b: a })
      const result = await ctx.call('add', { a: square, b: b })
  
      // Convert result to string and store within database
      const str = `a squared plus b is ${result}.`
      await ctx.db.set('ID', str)

      // Read out of database and create a custom log out of it
      await ctx.log(await ctx.db.get('ID'))

      return { result: str }
    }
  )
\end{minted}
\end{tcolorbox}
\caption{%
  A simple function showcasing all library capabilities within an rpcHandler.
  We expect that there are two other deployed functions `add' and `multiply',
  which return the sum and product of two numbers \mintinline{js}{a} and \mintinline{js}{b}, respectively.
}%
\label{fig:rpcHandlerFullFunctionalityExample}
\end{figure}

\begin{figure}
\begin{tcolorbox}
\begin{minted}{js}
  const lib = require('@faastermetrics/lib')
  module.exports = lib.serverless.router(
    { db: 'redis' }, async router => {
      router.post('/call', async (ctx, next) => {
        const { a, b } = ctx.request.body
    
        // Calling other functions to do calculations
        const square = await ctx.lib.call('multiply', { a: a, b: a })
        const result = await ctx.lib.call('add', { a: square, b: b })
   
        // Convert result to string and store within database
        const str = `a squared plus b is ${result}.`
        await ctx.lib.db.set('ID', str)
  
        // Read out of database and create a custom log out of it
        await ctx.lib.log(await ctx.db.get('ID'))
  
        ctx.type = 'application/json'
        ctx.body = { result: str }
      }
    }
  )
\end{minted}
\end{tcolorbox}
\caption{%
  The same functionality as in \Cref{fig:rpcHandlerFullFunctionalityExample}, 
  but using the router function wrapper instead of the rpcHandler.
}%
\label{fig:routerFullFunctionalityExample}
\end{figure}

\end{document}

