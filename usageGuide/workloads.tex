\documentclass[../main.tex]{subfiles}
\begin{document}


\section{Workloads}\label{sec:WorkloadsUsage}

\subsection{Setting up an experiment for workload deployment}

As outlined in \Cref{sec:WorkloadsStructure} we are using artillery to run workloads on experiments. 
In~short you add a workload to an experiment by placing a \texttt{<workload\_name>.yml} 
within the \texttt{experiments/<experiment\_name>} 
folder and editing the \texttt{experiment.json} to reference that workload:

\begin{tcolorbox}[titleDetachedStyle, title=\texttt{experiment.json}]
\begin{minted}{js}
...
  "services": {
    "workload": {
      "config": "./workload.yml"
    }, ...
...
\end{minted}
\end{tcolorbox}

TODO: add ref to full experiment.json here

\subsection{Defining your workload}

See the \texttt{official artillery documentation}\footnote{\url{https://artillery.io/docs/}} for how to structure your workloads. 
Be sure to set the \texttt{processor}, \texttt{beforeRequest} and \texttt{afterResponse} as in the example \Cref{fig:exampleWorkloadYML}. 
It is important that the \texttt{beforeRequest} and \texttt{afterResponse} are set for every single request.

In order to call a function with an rpcHandler, your request in artillery should look similarly to this:
\begin{tcolorbox}[titleDetachedStyle, title=\texttt{workload.yml}]
\begin{minted}[autogobble]{yaml}
  - post:
    url: '{{ functioname }}/call'
    json:
      arg1: 'blubb' 
      arg2: 'blabb'
    beforeRequest: 'beforeRequest'
    afterResponse: 'afterResponse'
\end{minted}
\end{tcolorbox}
In this example, \mintinline{yaml}{'{{ functioname }}'} will automatically resolve to the endpoint of the function \texttt{functionname}. \\
See \Cref{fig:exampleWorkloadYML} for a full example of a minimal \texttt{workload.yml}.

\subsection{Executing your workload}

After the workload is configured and your experiment is deployed you can run \texttt{fmctl workload <experiment\_name>} 
within the docker container. This will automatically provision a vps on AWS, run your workload on it, and
fetch its logs back to your computer for later analysis. Afterwards, the server-instance will be destroyed.
If you are not using docker, the workload command is \texttt{npm run workload <experiment\_name>}.

\subsection{Extending artillery}

You may wish to extend artillery functionality by JavaScript functions.
In this case, you have to edit the \texttt{artillery/logger.js} file as outlined in the artillery documentation%
\footnote{\href{https://artillery.io/docs/http-reference}%
{\texttt{https://artillery.io/docs/http-reference/\#advanced-writing-custom-logic-in-javascript}}}.


\end{document}
