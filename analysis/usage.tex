\documentclass[../main.tex]{subfiles}
\begin{document}

\section{Data Analysis Setup}

\begin{figure}
  System software requirements:
  \begin{enumerate}
    \item python 3.8 or later, with working pip support
    \item graphviz 2.40 or later
  \end{enumerate}
  Python package requirements:
  \begin{enumerate}
    \item seaborn
    \item pygraphviz
    \item networkx *
    \item argmagic *
    \item json\_coder *
  \end{enumerate}
  \caption[Required Dependencies for the Analysis Tools]%
  {Required Dependencies for the Analysis Tools. 
  Packages marked with * should already be installed with the FaasterMetrics package setup.}%
  \label{fig:analysisRequiredSetupTools}
\end{figure}


The analysis processing tools require a working installation of python 3.8, 
as well as the ability to install additional python packages via pip. 
For graph processing and visualizations graphviz will need to be installed on the system, 
either via a system package manager or manually. 
The full dependency list is shown in \Cref{fig:analysisRequiredSetupTools}

To get those dependencies, you may choose to employ a convenient setup via our provided dockerfile (\Cref{sub:analysisDockerSetup}),
use conda (\Cref{sub:analysisCondaSetup}, a respective environment file is also provided) or setup manually (\Cref{sub:analysisManualSetup}).
Whichever way you prefer, you can find everything you will need in our analysis code repository%
\footnote{\url{https://github.com/FaaSterMetrics/analysis}}.

\subsection{Installation via Docker setup}%
\label{sub:analysisDockerSetup}

This requires a working docker setup on the local system. 
Make sure the executing user has proper permissions for calling the docker daemon.
First build a base image with the name \texttt{fmanalysis}, or replace accordingly, in the analysis root directory:
\begin{tcolorbox}
  \mintinline{bash}{docker build -t fmanalysis}
\end{tcolorbox}\noindent
Afterwards this image can for example be used for executing the scripts interactively by starting a container with a mounted output directory:
\begin{tcolorbox}
  \mintinline{bash}{docker run -it fmanalysis -v output:/analysis/output}
\end{tcolorbox}

\subsection{Installation via Conda setup}%
\label{sub:analysisCondaSetup}

Conda allows for simple installation and management of an environment, which can contain alternative python versions. 
This allows for easier deployment on systems where a compatible python version is not included in the base system or system package repositories. 

Refer to the official conda documentation for setting up a basic conda installation%
\footnote{\url{https://docs.conda.io/en/latest/miniconda.html}}. 
Miniconda refers to a minimal version of conda without additional scientific packages. 
This version is sufficient for usage with our documentation.

After installation of conda, the \texttt{environment.yml} in our analysis repository can be used to set up a new conda environment:
\begin{tcolorbox}
  \mintinline{bash}{conda env create -f environment.yml}
\end{tcolorbox}\noindent
This will create a new environment with the name \texttt{faastermetrics2} (assumed for following commands). 
This name can be changed in the \texttt{environment.yml} file via its name entry.

In order to work inside the newly installed environment, it needs to be activated:
\begin{tcolorbox}
  \mintinline{bash}{conda activate faastermetrics2}
\end{tcolorbox}\noindent
Now the faastermetrics library can be installed via:
\begin{tcolorbox}
  \mintinline{bash}{python setup.py install}
\end{tcolorbox}\noindent
Alternatively, for a development install use:
\begin{tcolorbox}
  \mintinline{bash}{python setup.py develop}
\end{tcolorbox}\noindent
This will install the library into the conda environment. 
\texttt{graphviz} still needs to be installed on the host system, 
as conda installed \texttt{graphviz} might not work depending on local configuration.

\subsection{Direct local setup}%
\label{sub:analysisManualSetup}

This assumes a supported python implementation and graphviz library have been installed through other means. 
Installing third-party python libraries through pip should also be supported. 
Based on the usage requirements the FaasterMetrics library can either be installed directly or in development mode. 

For direct installation execute \texttt{python setup.py install} in the analysis base directory. 
Conversely \texttt{python setup.py develop} can be used for a development install. 
The latter allows changes to the library to be directly reflected in executed code without reinstallation,
thus being preferred for development usage.
\texttt{seaborn} and \texttt{pygraphviz} are only needed as dependencies for specific visualisation scripts 
and need to be installed manually, such as via pip:
\begin{tcolorbox}
  \mintinline{bash}{pip install seaborn pygraphviz}
\end{tcolorbox}

\section{Basic Analysis Tools Usage}

The analysis repository\footnote{\url{https://github.com/FaaSterMetrics/analysis}}
contains scripts for performing roughly the following actions:
\begin{enumerate}
  \item Log import (process raw experiment output data into analysis intermediary format)
  \item Export clean log data (i.e.\@ as CSV)
  \item Additional processing, such as grouping log entries based on function and request
  \item Visualization of call structure and basic performance comparisons
\end{enumerate}
After log import, the intermediary data version is used for all other procedures. 
The~following sections will elaborate on the aforementioned actions.

\subsection{Experiment Log Import}%
\label{sub:analysisUsageLogImport}

Logging output from experiments should be saved according to --TODO:\@ redacted--.
\texttt{scripts/dump\_logs.py} can be used to extract relevant log entries and store these into a structured .json file.
In order to just dump all logs from an experiment's log output folder into a single \texttt{logs.json} execute:
\begin{tcolorbox}
  \mintinline{bash}{./scripts/dump_logs.py <experiment_logs> ./logs.json}
\end{tcolorbox}\noindent
The generated json file will now contain all log entries generated by the experiment framework in individual objects.

If the input experiment log folder contains a \texttt{deployment\_id.txt}, 
this ID within will be used to automatically remove any log entries without the same ID. 
This should be sufficient to ensure that log entries are compatible and from the same experiment in most cases.

In some situations the raw experiment output might still contain entries from multiple experiment runs or
older entries with incompatible log format. 
For these purposes \texttt{dump\_logs} supports filtering loggging entries both on lib version and a time window:

\begin{tcolorbox}
  \mintinline{bash}{./scripts/dump_logs.py --version 7.0.1 <experiment_logs> ./logs.json}
\end{tcolorbox}\noindent
With this, only include log entries with a lib version of 7.0.1 are included. 
This can be useful for removing entries from incompatible lib versions.
\begin{tcolorbox}
  \mintinline{bash}{./scripts/dump_logs.py --timewindow 1h <experiment_logs> ./logs.json}
\end{tcolorbox}\noindent
With this, all entries younger than 1 hour from the most recent entry in the experiment logs are included.

The convenience script \texttt{./scripts/list\_logs.py} can be used on the raw experiment output folder 
to discover valid experiment folders and contained lib versions.

\subsection{Export to CSV}%
\label{sub:analysisUsageCSVExport}

Continuing from a log dumped to a local \texttt{logs.json}, \texttt{scripts/export.py} allows for generating a CSV file, 
which contains each log entry as a separate line that can be used for the analysis of logs in third-party tools.
In order to the export a \texttt{json} intermediary into a \texttt{logs.csv} the following line can be used:
\begin{tcolorbox}
  \mintinline{bash}{./scripts/export.py ./logs.json ./logs.csv}
\end{tcolorbox}\noindent

Accordingly, the complete steps in order to convert an raw experiment output into a CSV could be as follows:
\begin{tcolorbox}
\begin{minted}{bash}
./scripts/dump_logs.py <experiment_logs> ./logs.json
./scripts/export.py ./logs.json ./logs.csv
\end{minted}
\end{tcolorbox}

\subsection{Obtaining Context IDs}%
\label{sub:analysisUsageContextIDs}

The script \texttt{analysis\_function\_tree.py} performs a basic call graph analysis on a given logfile
and outputs function structure and context IDs in textual form. 
This can for example be used for specific plotting of context ID call graphs.
An example output from our webshop demo application is shown in \Cref{fig:contextIDExtractExample}.

\begin{figure}
\begin{tcolorbox}
  \begin{minted}[autogobble]{js}
  Keep with context ids only: 208/209
  = Calltree induced on: frontend =
  frontend: calls: 34 : {
    'c7q2dkyh', '81yfer69', 'oczsa428', 'pxnt022y', 'hf7dh8ms', 
    'qi33oz6w', 'uo63pgyq', 'miwpan1j', 'f5b9e4cu', 'fngs9lsy', 
    'vbt14juu', 'whgikujo', 'd2w0dn2o', 'z134z9v9', 't2e4jgl8', 
    'z116208c', 's7i5zszq', '2ycw9he5', 'slqlq3r5', '8kx8ogib', 
    'ukjlfqrt', '4n1791ob', 'fwqlcc4j', 'fuii60dd', 'yt8msp7u', 
    'ubk7slf2', 'tc9rpims', 'fbpyvyeb', 'q56ouw0u', '28yq2d0f', 
    'tiaei0k0', 'urirfiab', '77rui7sx', 'co5hxyoy'}
    supportedcurrencies: calls: 24 :
    listproducts: calls: 18 :
    getads: calls: 18 :
    currency: calls: 49 :
  [...]
  \end{minted}
\end{tcolorbox}
\caption[Function Tree Script Example]{%
  Shorted example output of the \texttt{analysis\_function\_tree.py} script for 
  an experiment run of our webshop demo application.
  Calls and their respective context IDs can be user for future analysis this way.%
}
\label{fig:contextIDExtractExample}
\end{figure}

\subsection{Visual Analysis and Processing}%
\label{sub:analysisUsageVisualProcessing}

The \texttt{plots} directory contains two visualization scripts, 
which can be used to obtain both a per-function execution time overview and 
a graph of either function calls with average execution times or individual call trees based on parsing of context IDs or X-Pairs TODO:\@ ref.

\texttt{plots/execution\_time.py} only takes two arguments, one for the input json 
log dump data and its output directory (where generated plots will be stored).
Example usage:
\begin{tcolorbox}
\begin{minted}{bash}
python plots/execution_time.py ./logs.json ./output/plots
\end{minted}
\end{tcolorbox}\noindent
With this, you generate a set of execution time output plots in the output directory \texttt{output/plots}. 
Characteristics of the individual visualisations will be discussed in TODO ref:executiontime.

\texttt{plots/function\_graph.py} provides graphviz-based visualisations 
of the graph model generated by the faastermetrics analysis library.
The standard functionality will generate a function graph with calls as edges and functions as nodes. 
Average execution times for both calls and functions are shown.
Example usage:
\begin{tcolorbox}
\begin{minted}{bash}
python plots/function_graph.py ./logs.json ./output/plots
\end{minted}
\end{tcolorbox}\noindent


The scripts per-default provides two different styles, either `classic' (default) or `modern'. 
These change the appearance of the generated graph and can be modified within the scripts, 
by changing the \texttt{STYLES} dictionary accordingly. 
For selecting the style, just use a respective command-line option:
\begin{tcolorbox}
\begin{minted}{bash}
# classic default style plots
python plots/function_graph.py --style classic ./logs.json 
./output/plots

# generate modern style plots
python plots/function_graph.py --style modern ./logs.json 
./output/plots
\end{minted}
\end{tcolorbox}

The option `--notime` removes time labels from the graph. 
A clean function graph in modern style without time labels is thus generated by the following:
\begin{tcolorbox}
\begin{minted}{bash}
python plots/function_graph.py --style modern --notime ./logs.json 
./output/plots
\end{minted}
\end{tcolorbox}

%These options only change the visual styling of the graphviz plotting, 
%the content presentation remains the same. 
Multiple other options are provided to change the function tree presentation. 
These include the ability to filter on specific function nodes or specific context IDs.

The options \texttt{--ftree}, \texttt{--degree}, and \texttt{--functions} include/exclude nodes based on specific criteria.

\texttt{--ftree} accepts a function name as a parameter and will print a function tree including all nodes connected with the given function.
In order to generate a function graph containing only nodes related with `cartkvstorage` the following could be used:
\begin{tcolorbox}
\begin{minted}{bash}
python plots/function_graph.py --ftree cartkvstorage ./logs.json 
./output/plots
\end{minted}
\end{tcolorbox}

\texttt{--degree} excludes nodes with degree equal or lower than the given value. 
This can be useful to debulk a very large function graph.
The following will remove all leaf nodes in our function graph:
\begin{tcolorbox}
\begin{minted}{bash}
python plots/function_graph.py --degree 1./logs.json ./output/plots
\end{minted}
\end{tcolorbox}



\texttt{--functions} can be used to explicitly list a number of functions to be included. 
Only these nodes will be used to contruct the graph. Edges will only be drawn, 
if both nodes are in the graph. 
No inducing behavior is performed for nodes with some higher degree of separation.

The following generates a graph containing only the functions `frontend' and `checkout'.
\begin{tcolorbox}
\begin{minted}{bash}
python plots/function_graph.py --functions '[frontend,checkout]' 
./logs.json ./output/plots
\end{minted}
\end{tcolorbox}

As an alternative to building the graph based on function names, 
which merge calls from different requests, 
the graph can also be used to visualize entirely based on individual calls. 
For this functionality a specific context ID needs to be given.

As an example, the following would generate the specific call tree for context ID 123456:
\begin{tcolorbox}
\begin{minted}{bash}
python plots/function_graph.py --context 123456 ./logs.json 
./output/plots
\end{minted}
\end{tcolorbox}











\end{document}
