\documentclass[../main.tex]{subfiles}
\begin{document}

As part of our project, we implemented two fully instrumented FaaS based demo applications:
An e-commerce application (i.e.\@ a webshop) and an IoT-resembling traffic light calculator for some road crossing.
They serve both as examples how to build fully working and adaptively deployable applications 
within our framework as well as a baseline for real application-oriented FaaS benchmarks.

\section{E-Commerce Application (Webshop)}\label{sec:webshop}

% Design Concepts
We implemented a test e-commerce application that fully consists of FaaS components together with a redis database. 
The basic functionality is to display recommended products and ads to customers 
who can then buy said products by charging a mocked credit card. 
As such, the deployed application can theoretically be ``used'' exclusively by humans 
interacting with its website frontend within a web browser.
Naturally, there are also pre-defined automized workloads to secure a benchmark's reproducibility.

The whole functionality is closely oriented on Google’s microservices demo v0.1.0\footnotemark 
but has of course been fully reimplemented as a FaaS project (except some HTML templates as referenced within their code directory). 
\footnotetext{\url{https://github.com/GoogleCloudPlatform/microservices-demo/tree/bae651f7ea537d2676b38a04d89adacdd45c17bd}}

\subsection{Application Structure}\label{ssec:webshopApplicationStructure}


\subsection{Benchmarking Properties}\label{ssec:webshopApplicationProperties>}


\subsection{Specific Workloads}\label{ssec:webshopSpecificWorkloads>}


\end{document}
