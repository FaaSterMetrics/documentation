\documentclass[../main.tex]{subfiles}
\begin{document}
\section{Intro}\label{sec:Intro}







\begin{longtable}{l l} 
  \caption[FaaS Benchmarking Suite Requirements]{Our FaaS Benchmarking Suite Requirements\vspace*{1mm}}\label{tab:requirements}\\
  \textbf{Requirement} & \textbf{Rationale}\\ 
  \toprule
  %\endhead{}%
  \makecell[{{p{4cm}}}]{%
  Application-Driven Benchmarking
  }&
  \makecell[{{p{10cm}}}]{%
    One main motivation for this project is that all current FaaS benchmarks we know of
    are microbenchmark-focused and do not completely represent the setting of real-world applications.
  }\\
  \midrule[0.02em]
  \makecell[{{p{4cm}}}]{%
  Support for Cross-Provider Setups
  }&
  \makecell[{{p{10cm}}}]{%
    When building complex real-world applications, it may be useful or needed to split
    functionality across multiple FaaS platforms.
    Additionally, our framework should also be able to evaluate those platforms themselves in terms of performance,
    given a specific scenario.
  }\\
  \midrule[0.02em]
  \makecell[{{p{4cm}}}]{%
  Easy Adaptibility to New Providers
  }&
  \makecell[{{p{10cm}}}]{%
    It should be possible to extend the framework, 
    such that support for novel and even experimental providers can later be added. %TODO achieved by e.g. common experiment process
    Especially in a fairly young environment such as the FaaS landscape,
    there may still emerge new platforms which are e.g.\@ specialized in a certain context.
  }\\
  \midrule[0.02em]
  \makecell[{{p{4cm}}}]{%
  External Database Support
  }&
  \makecell[{{p{10cm}}}]{%
    Since FaaS programming is stateless by its paradigm,
    interacting with external databases that keep any state is 
    inevitable in many FaaS applications.
    This need has to be supported by our framework.
  }\\
  \midrule[0.02em]
  \makecell[{{p{4cm}}}]{%
  Automated Build and Deployment Process %%% Note: Once set up
  }&
  \makecell[{{p{10cm}}}]{%
    Once set up and configured, no manual intervention should be needed 
    to get a program and its workload run.
  }\\
  \midrule[0.02em]
  \makecell[{{p{4cm}}}]{%
  Minimal Experiment Configuration Necessary %%% One File, declarative deployment, simple policy changes
  }&
  \makecell[{{p{10cm}}}]{%
    An experiment should be described with a single configuration file.
    It should be deployable in a declarative way, such that for a simple policy changes
    like moving a function from one platform to another one only single configuration edits are necessary
    while all remaining work is abstracted.
  }\\
  \midrule[0.02em]
  \makecell[{{p{4cm}}}]{%
  Time Measurements %%% Note: Function and Calling
  }&
  \makecell[{{p{10cm}}}]{%
  }\\
  \midrule[0.02em]
  \makecell[{{p{4cm}}}]{%
  Request Tracing 
  }&
  \makecell[{{p{10cm}}}]{%
  }\\
  \midrule[0.02em]
  \makecell[{{p{4cm}}}]{%
  Coldstart Detection %%% 
  }&
  \makecell[{{p{10cm}}}]{%
  }\\
  \midrule[0.02em]
  \makecell[{{p{4cm}}}]{%
  Data Export to Common Format (CSV) %%% 
  }&
  \makecell[{{p{10cm}}}]{%
  }\\
  \midrule[0.02em]
  \makecell[{{p{4cm}}}]{%
  Simple Data Visualisation %%% Common interests like function time + call graph, technically optional
  }&
  \makecell[{{p{10cm}}}]{%
  }\\
  \bottomrule
\end{longtable}


\end{document}

